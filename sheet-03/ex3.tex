% Template: Fabian Wenzelmann, 2016 - 2019
\documentclass[a4paper,
twoside, % to have to sided mode
headlines=2.1 % number of lines in the heading, increase if you want more
]{scrartcl}
\usepackage[
margin=2cm,
includefoot,
footskip=35pt,
includeheadfoot,
headsep=0.5cm,
]{geometry}
\usepackage[utf8]{inputenc}
\usepackage[english]{babel}
\usepackage[T1]{fontenc}
\usepackage{mathtools}
\usepackage{amssymb}
\usepackage{lmodern}
\usepackage{amsmath}
\usepackage[automark,headsepline]{scrlayer-scrpage}
\usepackage{enumerate}
\usepackage[protrusion=true,expansion=true,kerning]{microtype}
\usepackage{tabto}
% only required for the example
\usepackage{lipsum}

\newcommand{\yourname}{Najji Hendo \and Armin Saur}
\newcommand{\headingname}{Najji Hendo, Armin Saur}
\newcommand{\lecture}{Information Retrieval}
\newcommand{\sheetnum}{3}
\author{\yourname}
\title{\lecture}
\subtitle{Exercise Sheet \sheetnum}

\pagestyle{scrheadings}
\setkomafont{pagehead}{\normalfont}
\lohead{\lecture\\\headingname}
\lehead{\lecture\\\headingname}
\rohead{Exercise Sheet \sheetnum}
\rehead{Exercise Sheet \sheetnum}


\begin{document}
\maketitle
\section*{Exercise 1}
Proof $\Theta(\log_2 d_i \forall i$ for galloping-search.

The number of exponential  steps is $\log_2 di$, because:\\
For computing upper border applies $j' \geq j_i$, $j' = 2^{\lceil\log_2 d_{j'}\rceil}$.\\
$O_{exp}(\log_2 d_i) = \begin{cases}
d_i > d_{j'} & or\\
d_i = d_{j_i}
\end{cases}$\\
So the exponential search for $B[j_i]$ is over the distance $d_{j'}$, so it always needs the same number of steps.\\

The binary search for the complete distance $d_i$ is always overt the area of $d_{j'}$, which is, like the exponential search, in $O_{bin}(\log_2 d_i)$. If $d_i = d_{j'}$ the binary search is not needed.\\
$\Theta(log_2 d_i) = \begin{cases} O(\log_2 d_i) & \text{full exponential and binary search}\\ 
\Omega(\log_2 d_i) & \text{only exonential search}
\end{cases}$ 


\section*{Exercise 2}
% \begin{align*}
\begin{itemize} 
\item Proof why
$\mathcal{O}(\sum\limits_{i=1}^{k} log_2 d_i )$ ist not correct  unsing counterexample:
\\
when the gaps are always one then $d_i = 1$
\begin{center}
$ \Longrightarrow \mathcal{O}(\sum\limits_{i=1}^{k} \log_{2}{d_i} ) = \mathcal{O}(\sum\limits_{i=1}^{k} \log_{2}{1} )= 0 $.
\end{center}
and the time complexity cannot be 0.
\\
\\
\noindent
\item Proof
$\mathcal{O}(k . \sum\limits_{i=1}^{k} \log_{2}{(n/k)} )$ ist not correct unsing counterexample:
\\
When  ($len(A) = len(B)) \Longrightarrow k = n$
\begin{center}
$ \Longrightarrow \mathcal{O}(k.\sum\limits_{i=1}^{k} \log_{2}{(n/k)} ) = \mathcal{O}(k.\sum\limits_{i=1}^{k} \log_{2}{1} )= 0 $.
\end{center}
and the time complexity cannot be 0.

\end{itemize}

\newpage

\section*{Exercise 3}

\newcommand{\RN}[1]{%
\textup{\uppercase\expandafter{\romannumeral#1}}%
}
$k.\log_{2}{(1+n/k)} \leq k +n$
\\ 
constraint := $k.\log_{2}{(1+n/k)}  -k -n = 0$
\\ 
\begin{center}
$\Longrightarrow L:= k.\log_{2}{(1+n/k)} + \lambda(k+n)$ \\
$ = k.\log_{2}{(1+n/k)} +\lambda k + \lambda n$ 
\end{center}
\begin{itemize}
\item $\frac{\partial L}{\partial k} =  \log_{2}{(1+n/k)} + k \frac{-n}{k^{2}} . \frac{1}{1 + \frac{n}{k}}+ \lambda $ \tab\tab $(\RN{1})$\\

\item $\frac{\partial L}{\partial n} =  \frac{k}{k^{2}} . k \frac{1}{1+ \frac{n}{k}} + \lambda = \frac{1}{1+ \frac{n}{k}} + \lambda $ \tab\tab $(\RN{2})$\\

\item $\frac{\partial L}{\partial \lambda} =  -k-n$ \tab\tab $(\RN{3})$ \\
\end{itemize}
from $(\RN{1}) \Longrightarrow$ $ \lambda =  - \log_{2}{(1+n/k)} + \frac{n}{k} . \frac{1}{1 + \frac{n}{k}} $
\newline
from $(\RN{2}) \Longrightarrow$ $ \lambda =  \frac{-1}{1 + \frac{n}{k}} $
\newline

\begin{center}
from $(\RN{1}) $ and $ (\RN{2})$:
\\ $ - \log_{2}{(1+n/k)} + \frac{n}{k} . \frac{1}{1 + \frac{n}{k}} $ = $\frac{-1}{1 + \frac{n}{k}} $
\\
$ - \log_{2}{(1+n/k)} + \frac{n}{k + n} $ = $\frac{-1}{1 + \frac{n}{k}} $

\end{center}

\end{document}
